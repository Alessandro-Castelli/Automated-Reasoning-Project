\documentclass{article}
\usepackage[utf8]{inputenc} 
\usepackage[T1]{fontenc} 
\usepackage{graphicx}  
\usepackage{amsmath} 
\usepackage{amssymb} 
\usepackage{amsfonts} 
\usepackage[hidelinks]{hyperref}
\usepackage{fancyhdr}

\begin{document}

\title{Report Automated Reasoning} 
\author{Alessandro Castelli - (147073)} 
\date{\today}

\maketitle
\thispagestyle{empty}  
\newpage
\tableofcontents
\setcounter{page}{1}
\pagenumbering{arabic} % Numerazione in numeri arabi
\pagestyle{fancy}
\fancyhf{} % Pulisce intestazioni e piè di pagina
\fancyfoot[C]{\thepage} % Centra il numero di pagina nel piè di pagina
\renewcommand{\headrulewidth}{0pt} % Rimuove la linea orizzontale dell'intestazione
\newpage

\section{Introduzione}
Il presente report si concentra sull'analisi delle soluzioni al problema assegnato, 
riguardante l'organizzazione di un torneo di calcio a 7. 
Il problema in questione presenta diverse sfide legate alla suddivisione ottimale dei giocatori in squadre, rispettando specifici vincoli e minimizzando le penalità assegnate alle diverse configurazioni di squadra.

\subsection{Problema: Torneo di calcio a 7}
\begin{quote}
    Consideriamo un insieme di \( n \) giocatori di calcio \( p_1, p_2, \ldots, p_n \).
    Supponiamo che n sia un multiplo di 7. 
    Ciascuno di loro può giocare come portiere (g), difensore (d), centrocampista (m), attaccante (f). 
    Ognuno di loro sceglie esattamente due di queste opzioni (dati di input).
    Il problema è quello di dividerli in k squadre dove ciascuna squadra ha esattamente un g, almeno un d, almeno un m e almeno un f. Un ruolo unico viene assegnato a ciascuno di loro tra uno dei due dichiarati in input.
    Se una squadra ha esattamente 3 d, 2 m e 1 f, la sua penalità è 0. Se una squadra ha esattamente 2 d, 3 m e 1 f, la sua penalità è 0. Decidi (sei libero di assegnare valori) penalità non nulle per tutte le altre conformazioni.
    Trova una soluzione che minimizzi la somma delle penalità.
    Viene fornito anche un elenco di vincoli del tipo differentteam (\( p_i, p_j\)) 
    in input che costringono \(p_i\) e \(p_j\) a essere in squadre diverse. 
\end{quote}
Il problema affrontato è un \texttt{COP (Constraint Optimization Problem)}. Per risolverlo, sono stati sviluppati due programmi che impiegano due tecniche di programmazione distinte: \textbf{programmazione per vincoli con MiniZinc} e \textbf{Answer Set Programming utilizzando il solver Clingo}.
\\
Entrambi gli approcci sono stati selezionati per la loro efficacia nella gestione di problemi complessi e vincolati, sebbene differiscano nelle metodologie e negli strumenti adottati per ottenere la soluzione ottimale. 
\\
Nelle sezioni seguenti, verranno fornite descrizioni approfondite di entrambi gli approcci.
Inoltre, sarà presentata un'analisi dei risultati ottenuti da ciascun metodo. 

\section{Modellazione del problema}
Prima di iniziare la fase di programmazione, ho proceduto con lo studio e l'analisi del testo. Di seguito riporto alcune assunzioni che ho fatto:

Le assunzioni fatte per la risoluzione del problema sono le seguenti:

\begin{itemize}
    \item I giocatori iniziali sono \( n \) e \( n \) deve sempre essere un numero multiplo di 7.
    \item Per le \( k \) squadre, non è necessario che ogni squadra abbia esattamente 7 giocatori; ogni squadra può avere un minimo di 4 giocatori (uno per ogni ruolo) e un massimo di \( n \).
    \item Il numero di squadre \( k \) deve essere scelto in modo tale da avere senso rispetto al numero totale di giocatori; altrimenti, il problema non è soddisfacibile.
    \item Ogni giocatore può ricoprire solo uno dei due ruoli selezionati per la sua assegnazione finale.
    \item Ogni squadra deve avere esattamente un portiere (g), almeno un difensore (d), almeno un centrocampista (m) e almeno un attaccante (f).
    \item È previsto un sistema di penalità per configurazioni di ruolo specifiche, dove determinate combinazioni non comportano penalità.
    \item Sono forniti vincoli che richiedono che alcune coppie di giocatori siano assegnati a squadre diverse.
\end{itemize}

\section{Soluzione Minizinc}
Nella seguente sezione troverai la spiegazione del modello Minzinc che è stato corutito per risolvere il problema.

\end{document}
